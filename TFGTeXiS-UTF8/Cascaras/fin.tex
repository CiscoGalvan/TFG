%---------------------------------------------------------------------
%
%                      fin.tex
%
%---------------------------------------------------------------------
%
% fin.tex
% Copyright 2009 Marco Antonio Gomez-Martin, Pedro Pablo Gomez-Martin
%
% This file belongs to the TeXiS manual, a LaTeX template for writting
% Thesis and other documents. The complete last TeXiS package can
% be obtained from http://gaia.fdi.ucm.es/projects/texis/
%
% Although the TeXiS template itself is distributed under the 
% conditions of the LaTeX Project Public License
% (http://www.latex-project.org/lppl.txt), the manual content
% uses the CC-BY-SA license that stays that you are free:
%
%    - to share & to copy, distribute and transmit the work
%    - to remix and to adapt the work
%
% under the following conditions:
%
%    - Attribution: you must attribute the work in the manner
%      specified by the author or licensor (but not in any way that
%      suggests that they endorse you or your use of the work).
%    - Share Alike: if you alter, transform, or build upon this
%      work, you may distribute the resulting work only under the
%      same, similar or a compatible license.
%
% The complete license is available in
% http://creativecommons.org/licenses/by-sa/3.0/legalcode
%
%---------------------------------------------------------------------
%
% Contiene la última página
%
%---------------------------------------------------------------------

% Ponemos el marcador en el PDF
\ifpdf
   \pdfbookmark{Fin}{fin}
\fi

\thispagestyle{empty}
\small

% Centrar vertical y horizontalmente solo la frase
\vfill
\begin{center}
    \emph{Qué suerte tengo de tener algo que hace que decir adiós sea tan difícil.} \\
    \emph{Alan Alexander Milne}
\end{center}
\vfill

% Dedicatoria al final
\vspace*{\fill}
\begin{center}
    Este trabajo fin de grado no es solo un proyecto, es el reflejo de años de esfuerzo, ilusión y crecimiento.
\end{center}

\newpage
\thispagestyle{empty}\mbox{}

\newpage

% Variable local para emacs, para  que encuentre el fichero maestro de
% compilación y funcionen mejor algunas teclas rápidas de AucTeX

%%%
%%% Local Variables:
%%% mode: latex
%%% TeX-master: "../Tesis.tex"
%%% End:
