\chapter*{Resumen}

\section*{\tituloPortadaVal}

La creación de un videojuego es la combinación homogénea de varias disciplinas para crear una experiencia interactiva. Entre estas disciplinas encontramos arte, sonido, programación y diseño. A la hora de hacer arte, el ser humano siempre se ha ayudado de herramientas fabricadas por él mismo y le ha ayudado a llevar a cabo las tareas de manera más eficaz y precisa.
A la hora de hacer un videojuego de cualquier tipo, el equipo detrás de la obra tiene que dejar clara dos ideas al jugador: el objetivo del juego y los obstáculos a los que se va a tener que enfrentar para alcanzarlo. Haciendo especial hincapié en que en el tipo de juegos analizados en este trabajo, el principal obstáculo serán enemigos. Para que estos supongan un desafío para el jugador deberán contar con una inteligencia artificial acorde al diseño del mismo.\\\\ 

El objetivo de este trabajo es desarrollar una herramienta cuyo propósito sea generar enemigos para videojuegos de plataformas 2D. Para ello se diseñará un catálogo de componentes para Unity que permitirá crear, sin tener conocimientos previos de programación, comportamientos de enemigos. Este catálogo de componentes es el resultado de nuestro trabajo de investigación y posterior abstracción de comportamientos de enemigos en juegos de plataformas 2D. 
\section*{Palabras clave}
   
\noindent Inteligencia Artificial, IA, Unity, Enemigo, Maquinas de Estado, Estado, Sensor

   


