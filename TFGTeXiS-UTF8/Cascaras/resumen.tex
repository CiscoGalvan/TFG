\chapter*{Resumen}

\section*{\tituloPortadaVal}
Los videojuegos han evolucionado hasta volverse cada vez más sofisticados, combinando diversas disciplinas, como arte, sonido, programación y diseño. Además, en todo momento debe estar claro el objetivo del videojuego y los obstáculos que se deben superar. Concretamente en los plataformas 2D, los enemigos representan el principal obstáculo para el jugador. Con dicha sofisticación el sector es cada vez más técnico y el diseño de enemigos más complejo provocando la necesidad de perfiles con conocimientos en otras áreas para poder diseñar enemigos. Para evitar ese problema, surge la necesidad de crear una herramienta accesible que permita diseñar enemigos sin la barrera técnica. Para abordar este problema, se desarrollará un catálogo de componentes para Unity que facilitará la creación de comportamientos de enemigos en juegos de plataformas 2D, basado en una abstracción de patrones de comportamientos identificados en este tipo de juegos.
\section*{Palabras clave}
   
\noindent Inteligencia Artificial, Unity, Enemigo, Maquinas de Estado, 2D

   


