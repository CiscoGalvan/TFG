%---------------------------------------------------------------------
%
%                            TeXiS_cover.tex
%
%---------------------------------------------------------------------
%
% TeXiS_cover.tex
% Copyright 2009 Marco Antonio Gomez-Martin, Pedro Pablo Gomez-Martin
%
% This file belongs to TeXiS, a LaTeX template for writting
% Thesis and other documents. The complete last TeXiS package can
% be obtained from http://gaia.fdi.ucm.es/projects/texis/
%
% This work may be distributed and/or modified under the
% conditions of the LaTeX Project Public License, either version 1.3
% of this license or (at your option) any later version.
% The latest version of this license is in
%   http://www.latex-project.org/lppl.txt
% and version 1.3 or later is part of all distributions of LaTeX
% version 2005/12/01 or later.
%
% This work has the LPPL maintenance status `maintained'.
% 
% The Current Maintainers of this work are Marco Antonio Gomez-Martin
% and Pedro Pablo Gomez-Martin
%
%---------------------------------------------------------------------
%
% Fichero que  contiene la portada y  primera hoja de la  tesis, en la
% que se vuelve a repetir el título.
%
% El contenido  de la  portada se configura  en un fichero  externo en
% Cascaras/cover.tex
%
%---------------------------------------------------------------------

%%%
% Gestión de la configuración
%%%

% Título de la portada
\def\tituloPortadaVal{\titulo}
\newcommand{\tituloPortada}[1]{
\def\tituloPortadaVal{#1}
}

% Título de la portada en ingles
\def\tituloPortadaEngVal{\titulo}
\newcommand{\tituloPortadaEng}[1]{
	\def\tituloPortadaEngVal{#1}
}

% Autor
\def\autorPortadaVal{\autor}
\newcommand{\autorPortada}[1]{
\def\autorPortadaVal{#1}
}

% Tutor
\def\directorPortadaVal{Tutor no definido. Usa
  \texttt{$\backslash$directorPortada}}
\newcommand{\directorPortada}[1]{
\def\directorPortadaVal{#1}
}


% Colaborador
\def\colaboradorPortadaVal{Colaborador no definido. Usa
	\texttt{$\backslash$colaboradorPortada}}
\newcommand{\colaboradorPortada}[1]{
	\def\colaboradorPortadaVal{#1}
}


% Fecha de publicacion
\def\fechaPublicacionVal{}
\newcommand{\fechaPublicacion}[1]{
\def\fechaPublicacionVal{#1}	
}

% Imagen de la portada (escudo)
\def\imagenPortadaVal{Imagenes/Vectorial/Todo}
\newcommand{\imagenPortada}[1]{
\def\imagenPortadaVal{#1}
}
\def\escalaImagenPortadaVal{1.0}
\newcommand{\escalaImagenPortada}[1]{
\def\escalaImagenPortadaVal{#1}
}
% Tipo de documento (TESIS, MANUAL, ...)
\def\tipoDocumentoVal{TESIS DOCTORAL}
\newcommand{\tipoDocumento}[1]{
\def\tipoDocumentoVal{#1}
}

% Institución
\def\institucionVal{}
\newcommand{\institucion}[1]{
\def\institucionVal{#1}
}

% Primer subtítulo de la segunda portada
\def\textoPrimerSubtituloPortadaVal{%
\textit{Memoria que presenta para optar al título de Doctor en Informática}  \\ [0.3em]%
\textbf{\autorPortadaVal}%
}
\newcommand{\textoPrimerSubtituloPortada}[1]{
\def\textoPrimerSubtituloPortadaVal{#1}
}

% Segundo subtítulo de la segunda portada
\def\textoSegundoSubtituloPortadaVal{%
\textit{Dirigida por el Doctor}  \\ [0.3em]
\textbf{\directorPortadaVal}
}
\newcommand{\textoSegundoSubtituloPortada}[1]{
\def\textoSegundoSubtituloPortadaVal{#1}
}

% ISBN
\newcommand{\isbn}[1]{
\def\isbnVal{#1}
}

% Copyright
\newcommand{\copyrightInfo}[1]{
\def\copyrightInfoVal{#1}
}

% Créditos a TeXiS
\newcommand{\noTeXiSCredits}{
\def\noTeXiSCreditsVal{}
}

% Explicación sobre impresión a doble cara
\newcommand{\explicacionDobleCara}{
\def\explicacionDobleCaraVal{}
}
%%%
% Configuración terminada
%%%


%%%
%% COMANDO PARA CREAR LAS PORTADAS.
%% CONTIENE TODO EL CÓDIGO LaTeX
%%%
\newcommand{\makeCover}{


% Ponemos el marcador en el PDF
\ifpdf
\pdfbookmark{Página de Título}{titulo}
\fi

%
%  MÁRGENES
%
% La maquetación de las páginas en LaTeX es bastante complicada en lo
% que se refiere a los márgenes. Por ejemplo, _siempre_ hay un
% desplazamiento de una pulgada hacia la derecha y hacia abajo, porque
% por razones de maquetación se cree que es necesario ese espacio. Si
% quieres (en una hoja impar, que son de las únicas que me he
% preocupado) escribir por encima de la primera pulgada, o hacia la
% izquierda, tienes que usar "márgenes" negativos.
% 
% Para poder seguir esto un poco, lo mejor es que ejecutes \layout con
% el paquete layout cargado, para ver una imagen...
%
% Los valores importantes en lo que se refiere al margen (horizontal,
% que es del único que me he preocupado) son:
%
%   - \hoffset : desplazamiento horizontal del "eje" de coordenadas. A
%   este valor hay que sumarle, irremediablemente, una pulgada. El
%   valor por defecto es 0 pt
%   - \oddsidemargin : "margen" izquierdo (en las páginas impares). El
%   texto principal de los párrafos comenzará en esa posición (es
%   decir, 1 pulgada + \hoffset + \oddsidemargin). El valor por
%   defecto es 22 pt
%   - \textwidth : longitud del texto (de los párrafos). El valor por
%   defecto son 360 pt.
%   - \marginparsep : separador de la parte derecha del texto
%   principal y el espacio a anotaciones en el margen (notas
%   marginales). El valor por defecto es 7 pt.
%   - \marginparwidth : ancho de la sección de notas marginales. El
%   valor por defecto es 106 pt.
%
% Fijate que las notas marginales NO necesariamente llegarán hasta el
% final del folio. La separación entre el extremo derecho de las notas
% al margen y el final del folio en realidad dependerá del tamaño de
% éste; no se especifica de ninguna manera. Esto significa que si
% quieres ajustar exactamente el margen derecho tienes que echar
% cuentas con respecto al tamaño del papel, que se mantiene en
% \paperwidth.
%
% En realidad, no sé de qué manera pero todos los valores están
% relacionados de algún modo, y un cambio en uno afecta a los demás de
% maneras bastante inverosímiles. Además, _sólo_ pueden cambiarse en
% el preámbulo (bueno... al menos \textwidth, \oddsidemargin se puede
% cambiar en otros sitios, pero no se pueden hacer muchas cosas sólo
% con aquellos que se pueden cambiar...).
%
% En general se recomienda que NO cambies los márgenes. Están elegidos
% por especialistas que saben de maquetación y que han estudiado en
% profundidad las mejores organizaciones. Por ejemplo, los cambios que
% hagas pueden alargar demasiado las líneas, o dejarlas demasiado
% cortas. Pero bueno, si aun así quieres cambiar los márgenes _a nivel
% global_ es preferible que uses el paquete geometry, cuya inclusión
% recibe los centímetros de márgen que quieres en cada lado, y él se
% encarga de hacer las cuentas para que queden así, porque tocar tanto
% valor es un infierno.
%
% Si quieres cambiar los márgenes momentáneamente, entonces lo mejor
% es hacer uso de un entorno tipo "lista" que permite tocar algunos
% contadores para ajustar las posiciones de los párrafos. Eso es
% precisamente lo que hace el entorno cambiamargen definido en
% TeXiS.sty
%
% El problema es que esos cambios son _relativos_ a los márgenes
% oficiales. Si quieres hacer un cambio drástico (como el que
% necesitamos en la portada, para que quede centrada), entonces es
% necesario echar cuentas con los contadores anteriores para realizar
% el desplazamiento adecuado en cada lado para conseguirlo.
%
% Para resumir, los valores usados en los márgenes eran:
% 1 pulgada + hoffset + oddsidemargin +
%          + textwidth +
% + marginparsep + marginparwidth + AJUSTE 
%                                          = paperwidth
%
% Lo que necesitamos en crear un entorno cambiamargen pasando los
% valores adecuados para que quede centrado. Para eso, hay que hacer
% cuentas, y eso es un tanto infierno en LaTeX a no ser que se incluya
% el paquete calc que permite notación infija de operadores. Por
% tanto, para que esto funcione hay que incluirlo.
%
% Para aclararnos, vamos a crear un par de longitudes para hacer las
% cuentas poco a poco. Además, lo primero es asumir que queremos
% _eliminar todos los márgenes_ y que los párrafos lleguen totalmente
% de lado a lado. Para eso, en la izquierda tendremos que restar la
% suma de los tres primeros valores (1 pulgada, \hoffset y
% \oddsidemargin).

\newlength{\cambioIzquierdo}
\setlength{\cambioIzquierdo}{1in + \hoffset + \oddsidemargin}
% Si te falla aquí, incluye el paquete calc en el preámbulo.
% 1in = 1 pulgada = 72.27 pt

% En la parte derecha hay que restar el "margen" visible total, que es
% el tamaño de la hoja restando el espacio hasta la izquierda del
% párrafo y su ancho. Aprovechamos que el primer espacio lo tenemos en
% \cambioIzquierdo.
\newlength{\cambioDerecho}
\setlength{\cambioDerecho}{\cambioIzquierdo + \textwidth}
\setlength{\cambioDerecho}{\paperwidth - \cambioDerecho}

% Ya casi está. Si hicieramos
%
% \begin{cambiamargen}{-\cambioIzquierdo}{-\cambioDerecho}
%    ...
% \end{cambiamargen}
%
% tendriamos párrafos que van de extremo a extremo de la hoja. Como
% eso es una exageración, restamos a cada longitud el márgen real que
% queremos dejar.
\newlength{\margenPortada}
\setlength{\margenPortada}{3.5cm}

\setlength{\cambioIzquierdo}{\cambioIzquierdo - \margenPortada}
\setlength{\cambioDerecho}{\cambioDerecho - \margenPortada}


%%%
% Portada
%%%


% Página sin cabeceras
\thispagestyle{empty}

\begin{cambiamargen}{-\cambioIzquierdo}{-\cambioDerecho}


% En la primera hoja no se entiende de páginas pares e impares
\newlength{\evensidemarginOriginal}
\setlength{\evensidemarginOriginal}{\evensidemargin}

\newlength{\oddsidemarginOriginal}
\setlength{\oddsidemarginOriginal}{\oddsidemargin}

\setlength{\evensidemargin}{0cm}
\setlength{\oddsidemargin}{0cm}

% Comienza el tex...
\begin{Large}
\mbox{ }
\vfill

% Cuidado: Ajustarlo si el título cambia...
\newlength{\longTitulo}
\settowidth{\longTitulo}{%
\textbf{Arquitectura y metodología para el uso }}
\begin{center}
\rule{\longTitulo}{.5mm}\\
\vskip 1cm
\textbf{\tituloPortadaVal}\\
\textbf{\tituloPortadaEngVal}
\vskip 0.7cm
\rule{\longTitulo}{.5mm}\\
\end{center}
\end{Large}

\vfill

\begin{center}
  \includegraphics[scale=\escalaImagenPortadaVal]{\imagenPortadaVal}
\end{center}

\vfill

\newcounter{lastyear}
\setcounter{lastyear}{\the\year}
\addtocounter{lastyear}{-1}

\begin{center}
  {\Large \textbf{\tipoDocumentoVal}}\\
  {\Large \textbf{Curso \thelastyear--\the\year}}
  
\end{center}

\vfill

\begin{large}
\begin{center}
{\Large \bfseries Autor}\\
\textbf{\autorPortadaVal}\\[0.5cm]
{\Large \bfseries Director}\\
\textbf{\directorPortadaVal}\\[0.5cm]
\textbf{\institucionVal}\\[1em]
%\textbf{\fechaPublicacionVal}
\end{center}
\end{large}

\vfill

\end{cambiamargen}

\newpage

% Página que en la cara de detrás del folio de la portada.
% Ponemos que el documento esta maquetado con TeXiS y la
% aclaración de que debe imprimirse a doble cara.
\thispagestyle{empty}
\mbox{ }
%\vfill%space*{4cm}
%\begin{small} 
%\begin{center}
%\ifx\noTeXiSCreditsVal\undefined
%  Documento maquetado con \texis\ v.\texisVer.
%\else
%\mbox{ }
%\fi
%\end{center}
%\end{small}
%\vspace*{2cm}
%\begin{small} 
%\begin{center}
%\ifx\explicacionDobleCaraVal\undefined
%\mbox{ }
%\else
%\noindent Este documento está preparado para ser imprimido a doble
%cara.
%\fi
%\end{center}
%\end{small}

%%%
% Segunda portada
%%%

\newpage

\thispagestyle{empty}

\mbox{ }

\begin{Huge}
\begin{center}
\tituloPortadaVal \\
\tituloPortadaEngVal
\end{center}
\end{Huge}

\vfill

\begin{large}
\begin{center}
\textoPrimerSubtituloPortadaVal
\mbox{ } \\ \mbox{ } \\ 
{\Large \bfseries Autor}\\
\textbf{\autorPortadaVal}\\[1.0cm]
{\Large \bfseries Director}\\
\textbf{\directorPortadaVal}\\[0.3cm]
\mbox{ } \\ \mbox{ } \\ 
\textoSegundoSubtituloPortadaVal \\ [0.3em]
\end{center}
\end{large}

\vfill

\begin{large}
\begin{center}
\textbf{\institucionVal}\\[0.2em]
    \mbox{ }  \\
\textbf{\fechaPublicacionVal}
\end{center}
\end{large}


\newpage
\thispagestyle{empty}
%\mbox{ }
%
%% Información del ISBN y copyright
%\vskip 13cm
%\ifx\copyrightInfoVal\undefined
%\mbox{ }
%\else
%Copyright \textcopyright\ \copyrightInfoVal
%\fi
%\vskip 3cm
%\ifx\isbnVal\undefined
%\mbox{ }
%\else
%ISBN \isbnVal
%\fi

} % \newcommand{\makeCover}

% Variable local para emacs, para que encuentre el fichero
% maestro de compilación
%%%
%%% Local Variables:
%%% mode: latex
%%% TeX-master: "../Tesis.tex"
%%% End:
