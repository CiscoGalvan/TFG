\chapter*{Conclusions and Future Work}
\label{cap:conclusions}
\addcontentsline{toc}{chapter}{Conclusions and Future Work}
\section*{Final Conclusions}
This tool has been consistently focused on the design of 2D enemies. The proposal arises as a response to the need to provide designers with a practical and accessible solution that allows them to configure complex behaviors without requiring advanced programming knowledge.\\
Through an architecture based on finite state machines, and complemented by a set of sensors, actuators, and emitters, a flexible, accessible, and scalable solution has been achieved.\\
During development, essential aspects such as movement, damage control, entity generation, collision handling, and animation linking have been covered. In addition, a manual aimed at non-technical users has been included, which reinforces the accessibility purpose of the tool.


\section*{Future Work}

Looking towards future iterations, several areas of expansion have been identified that will allow the tool to have more functionalities and therefore versatility:\\
\begin{itemize}
  \item \textbf{Evolutionary Maintenance:} Periodic reviews will be necessary to ensure compatibility with future versions of the Unity engine, especially in the face of possible modifications to its base API.
  \item \textbf{Implementation of Sound Sensors:} Incorporating a sound sensor capable of detecting auditory stimuli was proposed, which would expand the interaction possibilities of enemies within the game.
  \item \textbf{Sound Emitters:} Complementary to acoustic sensors, these components would allow the emission of auditory signals that trigger behaviors in other nearby entities, enabling new game dynamics.
  \item \textbf{Steering and Avoidance Behaviors:} Integrating evasion techniques and conscious environment navigation is proposed, which would give enemies the ability to avoid obstacles and move more realistically.
  \item \textbf{Memory of Previous Interactions:} An advanced functionality would consist of allowing enemies to store information about past player actions, adapting their responses based on recognized patterns.
\item \textbf{Coordination Between Enemies:} Another line of improvement is to allow communication and cooperation between multiple enemies, generating collective behavior patterns, such as group chases or synchronized attacks.
 \item \textbf{Development of an External Graphical Interface:} In the long term, it is suggested to design an independent visual interface that facilitates the selection and configuration of components through a specific window, thus optimizing the user experience.
\end{itemize}

These lines of work reflect the potential of the system as an active tool within the 2D video game development workflow. The modular structure and the orientation towards scalability will allow these improvements to be incorporated progressively, guaranteeing its usefulness in real projects for the design of complex enemies.