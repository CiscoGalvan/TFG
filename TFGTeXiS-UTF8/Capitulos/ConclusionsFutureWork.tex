\chapter*{Conclusions}
\label{cap:conclusions}
\addcontentsline{toc}{chapter}{Conclusions}

\chapterquote{The mind of the subject will desperately struggle to create memories where none exist.}{Rosalind Lutece, \textit{BioShock Infinite}}


Throughout this Final Degree Project, we have tackled the challenge of providing 2D game designers with a comprehensive and user-friendly tool for configuring enemy behaviors, eliminating the need to write code. Starting from an exhaustive review of the state of the art—including solutions based on finite state machines, visual scripting systems such as PlayMaker and Unreal’s Blueprints, and node-based visual editors—we identified the best practices in accessibility and modularity that would underpin our framework.

\section*{Summary of Results}

\subsection*{Architectural Design}
Five main modules were defined:
\begin{itemize}
  \item \textbf{Sensor:} Subtypes for area detection, collisions, distance and time. This module adopts the \emph{observer} pattern to broadcast events in a decoupled manner, simplifying the creation of dynamic transitions without embedding high-level logic.
  \item \textbf{Actuator:} Abstract classes representing actions (movement and entity spawning). Thanks to inheritance and configurable parameters, each actuator implements its own initialization, update, and destruction logic.
  \item \textbf{Finite State Machine (FSM):} Orchestrates safe transitions between states, preventing inconsistencies and enabling complex behaviors through the combination of sensors and actuators.
  \item \textbf{Animation:} Via the \texttt{AnimatorManager} class and a custom \texttt{Animator Controller}, parameters for speed, direction, and events (spawn, damage, death) are synchronized with animations, seamlessly linking AI logic to visual representation.
  \item \textbf{Damage:} Clear separation between \texttt{DamageEmitter} and \texttt{DamageSensor}, allowing damage application and reception in a modular fashion and triggering side-effects (animations, UI events) without circular dependencies.
\end{itemize}
This component-based architecture ensures extensibility, high cohesion, and low coupling, enabling new modules or variants to be introduced without altering the framework’s core.

\subsection*{Implementation and Environment}
The framework was implemented as a Unity 2022.3.18f1 LTS package, which includes:
\begin{itemize}
  \item \emph{Debugging tools:} Scene gizmos, contextual tooltips, and debug modes that help visualize data flow.
  \item \emph{Usage examples:} Demonstration scenes combining various sensors and actuators to validate state integrity and animation correctness.
  \item \emph{Player components:} Advanced movement, collision detection, enhanced jumping, health management, and ranged attack scripts to verify the framework’s applicability in a complete scenario.
\end{itemize}

\subsection*{User Validation}
Seven participants (students from the UCM’s Game Design and Game Development master’s programs) evaluated the tool via:
\begin{itemize}
  \item \textbf{Quantitative method:} SUS questionnaire yielding an average score of 75.7 (\(\sigma=16.1\)), above the usability benchmark of 68.
  \item \textbf{Observation:} Identification of bottlenecks when adding new states and low tooltip visibility.
  \item \textbf{Semi-structured interviews:} Feedback on the need for in-editor video tutorials, improved documentation clarity, and better visualization of node connections.
\end{itemize}
Despite the limited sample and single session, qualitative and quantitative data consistently highlight clear workflows and strong modularity, while pointing out areas for improving contextual help.

\section*{Contributions and Lessons Learned}

\begin{itemize}
  \item \textbf{Accessibility for non-programmers:} A well-structured visual configuration system can match the power of code-based solutions, allowing designers to focus on gameplay rather than syntax.
  \item \textbf{Importance of interactive documentation:} Extensive written manuals are no substitute for practical guides (videos, in-editor tutorials) in speeding up the learning curve.
  \item \textbf{Modularity as a maintenance cornerstone:} Designing clear, self-contained modules enables the framework to evolve and integrate new features without regressions.
  \item \textbf{Logic–visual synchronization:} Integrating FSM, sensors, actuators, and animations in a unified pipeline ensures coherence between AI logic and the user’s visual experience.
\end{itemize}

\section*{Future Work}

To consolidate and extend this development, the following action lines are proposed:

\begin{enumerate}
  \item \textbf{Enhanced contextual help and UX:}
    \begin{itemize}
      \item Redesign and surface more prominent, configurable tooltips.
      \item Create interactive tutorials embedded within the Unity editor.
    \end{itemize}
  \item \textbf{Expanded sensors and emitters:}
    \begin{itemize}
      \item Add a sound sensor and acoustic emitters for behavior based on auditory stimuli.
      \item Incorporate multi-condition sensors (e.g., combining distance and collision triggers).
    \end{itemize}
  \item \textbf{Advanced navigation and AI:}
    \begin{itemize}
      \item Integrate \emph{steering} algorithms and obstacle avoidance.
      \item Develop a memory system for past interactions and inter-agent communication for cooperative behaviors.
    \end{itemize}
  \item \textbf{Graphical editing interface:}
    \begin{itemize}
      \item Implement a drag-and-drop node editor (states, transitions) with property panels.
      \item Provide filterable views and logical groupings to manage large state graphs.
    \end{itemize}
  \item \textbf{Maintenance and scalability:}
    \begin{itemize}
      \item Establish a periodic update plan to maintain compatibility with new Unity versions and API changes.
      \item Publish the framework as an official Unity package to encourage community adoption and contributions.
    \end{itemize}
\end{enumerate}

Overall, this project lays a solid foundation for an accessible, powerful, and scalable 2D enemy behavior framework, equipping both designers and programmers with tools that enhance creative efficiency and technical coherence in game development.
