\setcounter{secnumdepth}{3} %para tener una profundidad más en las enumeraciones
\chapter{Evaluacion Con Usuarios}
\label{cap:evaluacionConUsuarios}
Tras las fases de desarrollo de la herramienta se llevarán a cabo las pruebas con usuarios que nos mostrarán los datos de un primer contacto con usuarios.\\
Para eso hemos creado un guión de evaluación que fijará unos objetivos y unas preguntas de investigación, así como marcar las pautas para la realización de las pruebas. \\
Después de la realización de las pruebas se procederá a hacer una análisis que muestre las posibles brechas encontradas, basándonos en la experiencia de los usuarios. \\
\section{Objetivos y preguntas de investigación}
Los objetivos y preguntas de investigación tienen como finalidad tratar de comprender la experiencia de los usuarios reales al interactuar con nuestra herramienta por primera vez. Buscamos identificar cualquier punto donde puedan surgir dificultades, confusión o áreas donde la herramienta podría ser más efectiva. Para guiar esta exploración, hemos definido los siguientes objetivos y preguntas de investigación:

\subsection{¿Resulta intuitiva y clara la herramienta para el usuario?}
Lo más importante de la herramienta es que sea clara para los usuarios que van a usarla, ya que sin eso, aunque la herramienta tenga un buen funcionamiento, la utilidad de esta sigue siendo nula.\\
Nuestro primer objetivo se centra en evaluar si los usuarios perciben la interfaz y los flujos de trabajo de manera natural y sin ambigüedades.\\

Preguntas de investigación relacionadas con el objetivo:
\subsubsection{¿El usuario entiende fácilmente cómo añadir funcionalidad al enemigo?}
Esta pregunta de investigación pretende comprobar, si el usuario es capaz de comprender el funcionamiento base de los enemigos, es decir, si comprende como utilizar la máquina de estados o como se añaden estados nuevos, transiciones, Damage Emitters y actuadores.\\

\subsubsection{¿Las opciones de comportamiento ofrecidas son comprendidas en su totalidad?}
Aquí, nuestro interés reside en la claridad con la que se presentan y se entienden los distintos actuadores disponibles. Buscamos identificar cualquier punto de confusión en su aplicación, analizando si la funcionalidad y las explicaciones proporcionadas para cada uno son suficientes para su correcta utilización.\\

\subsubsection{¿La interfaz comunica correctamente el propósito de cada estado y transición?}
La pregunta de investigación pretende conocer si el diseñador es capaz de saber en todo momento en que estado está el enemigo y por tanto que comportamiento cabría esperar.\\

\subsubsection{¿La documentación aclara y detalla la configuración de las funcionalidades?}
Buscamos evaluar si la documentación adjunta proporciona la información necesaria para que el usuario pueda configurar correctamente los enemigos, aprovechando al máximo las funcionalidades implementadas. \\

\subsection{¿El sistema de creación de enemigos demuestra ser funcional?}
Este objetivo busca validar la operatividad y la utilidad de la herramienta.\\

Preguntas de investigación relacionadas con el objetivo:
\subsubsection{¿La herramienta agiliza y simplifica el proceso de diseño de enemigos?}
Esta pregunta se enfoca en la eficiencia que aporta la herramienta al flujo de trabajo del diseñador de videojuegos. ¿Facilita la creación de enemigos?\\

\subsubsection{¿El comportamiento de los enemigos generados se corresponde con lo esperado en el videojuego?}
En este punto, la atención se centra en la fidelidad con la que la lógica definida en la herramienta se manifiesta en el entorno de juego. ¿Actúan los enemigos según lo previsto? ¿Se activan sus acciones en las circunstancias correctas? Identificar cualquier desviación entre el diseño y la ejecución es crucial.\\

\subsubsection{¿La herramienta permite la adaptación y expansión sencilla de los enemigos?}
Queremos evaluar si es posible modificar o ampliar las capacidades de los enemigos creados mediante la adición de nuevos comportamientos o la alteración de parámetros existentes, sin generar complejidad innecesaria.

