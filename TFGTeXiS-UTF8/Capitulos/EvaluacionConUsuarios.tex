\setcounter{secnumdepth}{3} %para tener una profundidad más en las enumeraciones
\chapter{Evaluacion Con Usuarios}
\label{cap:evaluacionConUsuarios}
Una vez finalizada la fase de implementación de la herramienta, se consideró fundamental validar su utilidad y usabilidad a través de pruebas con usuarios. Estas pruebas permiten obtener una primera impresión del sistema desde el punto de vista de quienes no han participado en el desarrollo, y proporcionan información clave sobre posibles mejoras o dificultades de uso.\\

Para ello, se diseñó un protocolo de evaluación estructurado que incluye objetivos claros, preguntas de investigación específicas y una guía detallada para la realización de las pruebas. El objetivo principal de esta fase es identificar brechas entre el diseño previsto y la experiencia real del usuario, así como recoger sugerencias que puedan orientar futuras iteraciones de la herramienta.\\

En las siguientes secciones se describen los objetivos y preguntas de investigación, el guion de evaluación utilizado, el desarrollo de las sesiones de prueba y el análisis de los resultados obtenidos.\\

\section{Objetivos y preguntas de investigación} \label{sec:preguntas}

En el contexto de esta evaluación, se ha definido un objetivo general: comprender la experiencia de uso de los usuarios al interactuar por primera vez con la herramienta. Este objetivo engloba aspectos clave como la facilidad de uso, la claridad de la interfaz y la capacidad del sistema para guiar al usuario en la configuración de comportamientos sin necesidad de escribir código.

A partir de este objetivo general, se han formulado dos líneas de evaluación principales, cada una con sus correspondientes preguntas de investigación, que servirán para guiar tanto la observación como la recogida de datos cualitativos.

\begin{description}
	\item[\textbf{1. Evaluar la claridad e intuición de la herramienta.}]
Esta línea se centra en determinar si los usuarios perciben la herramienta como fácil de entender y utilizar, incluso sin conocimientos técnicos. Lo más importante es que la interfaz y los flujos de trabajo resulten naturales, sin ambigüedades ni puntos de bloqueo.

	\begin{enumerate}
		\item \textbf{¿El usuario entiende fácilmente cómo añadir funcionalidad al enemigo?}\\
 Esta pregunta pretende verificar si el usuario comprende el funcionamiento base de los enemigos, incluyendo el uso de la máquina de estados, la creación de nuevos estados, la configuración de transiciones, actuadores y Damage Emitters.

		\item \textbf{¿Las opciones de comportamiento ofrecidas son comprendidas en su totalidad?}\\
 Se analiza si la funcionalidad de los distintos actuadores es clara para el usuario y si las descripciones disponibles en la herramienta son suficientes para usarlos correctamente.

		\item \textbf{¿La interfaz comunica correctamente el propósito de cada estado y transición?}\\
 Evaluamos si el usuario puede interpretar visualmente en qué estado se encuentra el enemigo en cada momento y cómo se espera que se comporte.

		\item \textbf{¿La documentación aclara y detalla la configuración de las funcionalidades?}\\
 Se analiza si el material de apoyo resulta suficiente y útil para ayudar al usuario a entender la herramienta sin necesidad de asistencia externa.
	\end{enumerate}

	\item[\textbf{2. Validar la funcionalidad y utilidad práctica del sistema.}]
 Esta línea tiene como objetivo verificar que la herramienta no solo es comprensible, sino que también es útil y funcional en el flujo de trabajo de diseño de videojuegos.

	\begin{enumerate}
		\item \textbf{¿La herramienta agiliza y simplifica el proceso de diseño de enemigos?}\\
 Se estudia si los usuarios sienten que la herramienta mejora la eficiencia de su proceso creativo respecto a métodos más tradicionales.

		\item \textbf{¿El comportamiento de los enemigos generados se corresponde con lo esperado en el videojuego?}\\
 Evaluamos si lo que el usuario configura en el editor se traduce correctamente al comportamiento observado en el entorno de juego.

		\item \textbf{¿La herramienta permite la adaptación y expansión sencilla de los enemigos?}\\
 Esta pregunta busca determinar si es posible modificar o ampliar los enemigos existentes sin generar confusión ni complejidad innecesaria.
	\end{enumerate}
\end{description}

\section{Diseño de la Evaluación}
En esta sección se detalla la planificación y la estructura de las pruebas con usuarios.

\subsection{Audiencia Objetivo}
\begin{itemize}
\item Edad: mayor de 18 años
\item Genero: No relevante
\item Extracto Sociocultural: El público objetivo se centra en perfiles que tengan un extracto sociocultural relacionado con el diseño de videojuegos. Esto incluye:
\begin{itemize}
\item Estudiantes actuales o pasados de diseño de videojuegos.
\item Profesionales que se dediquen o tengan interés en la creación de enemigos.
\item Personas que sin poseer estudios específicos demuestren tener interés en la creación de entidades y sus comportamientos. 
\end{itemize}

\item Habilidades Requeridas:  Se espera que los participantes en la evaluación cuenten con conocimientos previos en diseño de personajes, conceptos básicos del funcionamiento de Unity y comprensión básica de maquinas de estado.
\end{itemize}

\subsection{Duración y Entorno de Realización}
Cada sesión de prueba se extenderá durante unos 60 minutos. Este lapso lo dedicaremos primero a una breve charla introductoria sobre la prueba. Luego, se pedirá que se explore el manual y realicen los ejemplos prácticos. Finalmente se realizará una entrevista que proporcionará feedback adicional.\\

Las sesiones se realizarán en entornos controlados y tranquilos donde el usuario no tenga distracciones. Se dispondrá un ordenador con la herramienta ya instalada y lista para su uso, así como teclado y ratón estándar. Al haber simultaneidad de pruebas con varios usuarios, para evitar la influencia entre ellos se procurará situarlos separados entre ellos y en caso de que no fuese posible, pedirles la mínima interacción posible entre ellos.

\subsection{Descripción de Tareas del Probador}
El probador deberá realizar las siguientes actividades:

\begin{itemize}
\item Lectura del manual (sin ejemplos de uso). Estimación: 10 minutos.
\item Ejecución de ejemplos de uso. Estimación: 30 minutos.
\item Creación libre de un enemigo. Estimación 10 minutos
\item Realización del cuestionario. Estimación 5 minutos.
\item Realización de la entrevista. Estimación 5 minutos.
\end{itemize}

Las tareas se presentarán de forma secuencial, permitiendo al usuario familiarizarse gradualmente con las funcionalidades de la herramienta. 

\subsection{Instrucciones Iniciales}

Antes de comenzar las pruebas se agradecerá la participación de los usuarios. Se dejará claro al inicio que no es una crítica hacia ellos y que solo estamos evaluando la funcionalidad de la herramienta y en ningún caso juzgándoles, además, se les indicará las diferentes partes que consta la evaluación.\\

\subsection{Comportamiento del Investigador}

Durante toda la sesión de pruebas, el investigador actuará como observador imparcial y facilitador. Su presencia tendrá como objetivo principal garantizar que los participantes comprendan las instrucciones generales y ofrecer soporte logístico en caso de que surjan problemas técnicos o situaciones inesperadas. No se ofrecerá ayuda directa sobre el uso de la herramienta salvo que una situación impida completamente continuar con la evaluación, ya que se busca observar cómo los usuarios interactúan con la interfaz sin intervenciones externas.\\

El investigador evitará cualquier influencia en las decisiones o acciones del usuario, no corrigiendo ni orientando sobre cómo utilizar la herramienta. Anotará observaciones relevantes sobre el comportamiento del usuario, como expresiones de duda, errores recurrentes, pasos omitidos o dificultades generales en la navegación.\\

Al finalizar la sesión, el investigador facilitará la entrevista, siguiendo una estructura abierta pero guiada que permita profundizar en las impresiones subjetivas de cada usuario.\\

\subsection{Metodología}

La evaluación ha sido diseñada utilizando una metodología mixta, combinando técnicas cuantitativas y cualitativas con el objetivo de obtener una visión completa del desempeño de la herramienta, tanto en términos de usabilidad como de experiencia percibida por los usuarios.

\subsubsection{Enfoque cualitativo}

El enfoque cualitativo busca recoger impresiones, percepciones y comportamientos de los usuarios durante la interacción con la herramienta, a través de técnicas como la observación directa y la entrevista semiestructurada.

\paragraph{Observación}
Durante la realización de las tareas, el investigador observará de forma no intrusiva el comportamiento de los participantes. Se tomarán notas sobre:

\begin{itemize}
\item Dudas frecuentes o repetitivas.
\item Tiempos estimados de realización por tarea.
\item Momentos de frustración, vacilación o errores.
\item Fluidez general en la interacción con la interfaz.
\end{itemize}

Esta observación permitirá identificar posibles problemas de usabilidad no detectables únicamente a través del cuestionario.

\paragraph{Puesta en común}
Tras la finalización del cuestionario SUS, se llevará a cabo una entrevista semiestructurada con cada participante. Esta dinámica de cierre busca profundizar en la experiencia del usuario, identificar áreas de mejora no previstas y recoger sugerencias espontáneas. 

A continuación, se muestran algunas de las preguntas que servirán como guía para esta conversación:

\begin{itemize}
\item ¿Hubo algo que os sorprendiera o no esperabais durante el uso de la herramienta?
\item ¿Qué parte del proceso de creación de enemigos os pareció más interesante o satisfactoria?
\item ¿En qué momentos sentisteis que no sabíais muy bien qué hacer o cómo proceder?
\item ¿Cambiaríais algo de la forma en la que se explican los elementos como estados, transiciones o actuadores?
\item ¿Creéis que la herramienta permite expresar bien ideas de diseño complejas?
\item ¿Consideráis que podríais usar esta herramienta para un proyecto real o profesional? ¿Por qué?
\item ¿Hay alguna funcionalidad que echasteis de menos mientras la usabais?
\item ¿Os surgieron ideas o sugerencias que podrían mejorar la herramienta?
\end{itemize}

\subsubsection{Enfoque cuantitativo}

El enfoque cuantitativo está centrado en recoger datos objetivos y medibles a través de un cuestionario estandarizado. En este caso, se ha utilizado el cuestionario SUS (System Usability Scale), desarrollado por Brooke (1996), con el fin de evaluar la percepción de usabilidad del sistema de forma estructurada y comparable.

\paragraph{Cuestionario SUS}
Una vez completadas las tareas, los participantes responderán al cuestionario SUS (System Usability Scale), propuesto por Brooke (1996). Este instrumento se compone de diez afirmaciones que deben valorarse mediante una escala de Likert de cinco puntos, donde 1 representa "totalmente en desacuerdo" y 5 representa "totalmente de acuerdo".

El cuestionario está diseñado para ofrecer una puntuación global de usabilidad del sistema. Sin embargo, en el contexto de esta evaluación, también se analizarán de manera individual algunas de las respuestas, ya que ciertas afirmaciones están directamente relacionadas con las preguntas de investigación planteadas previamente. De esta forma, se podrá extraer información más específica sobre aspectos concretos de la herramienta, como su facilidad de aprendizaje, consistencia o integración funcional.

Las afirmaciones que componen el cuestionario son las siguientes:

\begin{enumerate}
    \item Creo que me gustaría usar este sistema con frecuencia.
    \item Encontré el sistema innecesariamente complejo.
    \item Pensé que el sistema era fácil de usar.
    \item Creo que necesitaría la ayuda de una persona técnica para poder usar este sistema.
    \item Encontré que las diversas funciones de este sistema estaban bien integradas.
    \item Pensé que había demasiada inconsistencia en este sistema.
    \item Imaginaría que la mayoría de la gente aprendería a usar este sistema muy rápidamente.
    \item Encontré el sistema muy engorroso de usar.
    \item Me sentí muy confiado al usar el sistema.
    \item Necesité aprender muchas cosas antes de poder usar este sistema.
\end{enumerate}

Este cuestionario permitirá calcular una puntuación global de usabilidad, útil para comparar esta herramienta con otras o evaluar mejoras a lo largo del tiempo.

\section{Resultados de las pruebas}

\subsection{Contexto de las sesiones}

Las pruebas se llevaron a cabo en una única sesión de aproximadamente una hora de duración. Participaron siete voluntarios, todos ellos estudiantes del Máster en Diseño de Videojuegos o del Máster en Desarrollo de Videojuegos de la UCM, con conocimientos previos de diseño y, en algunos casos, de programación. Todas las evaluaciones se realizaron sobre la misma versión del proyecto; pese a que se utilizaron distintas versiones de Unity, no se detectaron problemas técnicos derivados de ello.

\subsection{Datos cuantitativos}

Para medir la usabilidad de la herramienta se empleó el cuestionario SUS (System Usability Scale) de Brooke (1996), que proporciona una puntuación de 0 a 100. La puntuación obtenida fue de \(\bar{X}=75{,}7\) (desviación estándar \(\sigma=16{,}1\)), lo que supera el umbral de 68 puntos considerado “por encima de la media” en la literatura de usabilidad. Aunque el tamaño de la muestra es limitado, este resultado indica una percepción global positiva de la herramienta.

\subsection{Datos cualitativos}

Mediante observación directa y entrevista semiestructurada se recogieron impresiones de los usuarios:

\begin{itemize}
  \item Dudas frecuentes sobre la existencia de tooltips (ningún participante los localizó espontáneamente).
  \item Confusión entre el estado de la máquina de estados y el estado de animación.
  \item Recomendaciones de gestionar las animaciones desde el Inspector en lugar de la pestaña \textit{Animator}.
  \item Sugerencia de incluir vídeos tutoriales para ilustrar la configuración de comportamientos.
  \item Valoración positiva de la gran variedad de comportamientos y de la modularidad del sistema.
  \item Necesidad de un feedback visual más claro en casos como la aplicación de daño y herramientas de depuración extendidas.
\end{itemize}

\section{Análisis de resultados}

A continuación se analiza cómo estos datos responden a las preguntas de investigación formuladas en la Sección~\ref{sec:preguntas}, combinando los tres métodos empleados (cuestionario SUS, observación y entrevista).

\subsection{Claridad e intuición de la herramienta}

\textbf{Objetivo:} Evaluar si la interfaz y los flujos de trabajo resultan naturales y sin ambigüedades.

\begin{itemize}
  \item \emph{Cuestionario SUS:}  
    \begin{itemize}
      \item Los ítems 3 (\textit{Pensé que el sistema era fácil de usar}) y 7 (\textit{Imaginaría que la mayoría de la gente aprendería a usar este sistema muy rápidamente}) reflejan que más del 85\,\% de los usuarios seleccionaron valores altos (4 o 5), lo que indica una percepción general positiva respecto a la facilidad de uso y aprendizaje inicial.
    \end{itemize}
  
  \item \emph{Observación:}  
    \begin{itemize}
      \item Se observaron tiempos adecuados de realización de las tareas, aunque algunos usuarios mostraron dudas concretas a la hora de añadir estados, transiciones o elementos como actuadores y emisores de daño.
    \end{itemize}
  
  \item \emph{Entrevista:}  
    \begin{itemize}
      \item Ninguno de los participantes hizo uso de los tooltips integrados, lo cual se debió a que no sabían que existían. En las entrevistas manifestaron que no los habían detectado visualmente. Se concluye que una posible mejora sería mencionar explícitamente su uso en el manual, ya que modificar su diseño gráfico podría sobrecargar visualmente la herramienta.
    \end{itemize}
\end{itemize}
\subsection{Funcionalidad y utilidad práctica}

\textbf{Objetivo:} Verificar si la herramienta agiliza y simplifica el proceso de diseño de enemigos, y si el comportamiento de los enemigos generados se corresponde con lo esperado dentro del videojuego.

\begin{itemize}
  \item \emph{Cuestionario SUS:}
    \begin{itemize}
      \item Este aspecto se refleja indirectamente en los ítems 1 (\textit{Creo que me gustaría usar este sistema con frecuencia}) y 3 (\textit{Pensé que el sistema era fácil de usar}). En ambos casos, la mayoría de los usuarios seleccionaron valores de 4 o 5, lo que sugiere una buena percepción de utilidad y eficiencia de la herramienta a la hora de diseñar enemigos.
    \end{itemize}
  
  \item \emph{Observación:}
    \begin{itemize}
      \item Durante las pruebas, se comprobó que los enemigos configurados se comportaban correctamente según los ejemplos incluidos en el manual. Sin embargo, algunos usuarios mostraron dificultades para comprender completamente el funcionamiento subyacente, lo que ralentizó ligeramente la ejecución de ciertas tareas.
    \end{itemize}
  
  \item \emph{Entrevista:}
    \begin{itemize}
      \item Los participantes sugirieron la incorporación de vídeos explicativos como complemento al manual actual. Consideraron que una guía visual podría facilitar el entendimiento del proceso de configuración y ayudar a relacionar mejor la lógica de edición con el comportamiento final observado en la escena.
    \end{itemize}
\end{itemize}

\subsection{Adaptación y extensibilidad}

\textbf{Objetivo:} Determinar si es sencillo modificar o ampliar los comportamientos de los enemigos sin causar confusión.

\begin{itemize}
  \item \emph{Cuestionario SUS:}
    \begin{itemize}
      \item Ítem 10 (“Necesité aprender muchas cosas antes de usarlo”) registró principalmente valores bajos (1–2), indicando curva de aprendizaje baja.
    \end{itemize}
  \item \emph{Observación:}
    \begin{itemize}
      \item Los usuarios probaron con éxito variaciones de parámetros sin guía externa.
    \end{itemize}
  \item \emph{Entrevista:}
    \begin{itemize}
      \item La modularidad recibió elogios, aunque se planteó mejorar la visualización de las conexiones entre estados y actuadores.
    \end{itemize}
\end{itemize}

En conjunto, el análisis muestra que la herramienta cumple su propósito de ofrecer una interfaz accesible y eficiente, si bien puede beneficiarse de mejoras en la visibilidad de ayudas contextuales (tooltips, tutoriales) y en la representación visual de las relaciones internas para reducir la curva de aprendizaje en configuraciones más complejas.
\subsection{Conclusión general}

La evaluación preliminar de la herramienta sugiere que la usabilidad es generalmente aceptable en su versión actual. Se perciben fortalezas en la claridad de la interfaz, la integración de funcionalidades y una baja percepción de complejidad. Sin embargo, es importante señalar que la muestra reducida de usuarios impide que estos datos sean estadísticamente significativos o generalizables.\\

Las observaciones y respuestas cualitativas de este pequeño grupo indican posibles áreas de mejora en la herramienta. A pesar de estas sugerencias, parece cumplir los objetivos definidos a un nivel básico para los usuarios que la han probado.

