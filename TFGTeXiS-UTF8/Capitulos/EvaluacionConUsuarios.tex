\setcounter{secnumdepth}{3} %para tener una profundidad más en las enumeraciones
\chapter{Evaluacion Con Usuarios}
\label{cap:evaluacionConUsuarios}
Tras las fases de desarrollo de la herramienta se llevarán a cabo las pruebas con usuarios que nos mostrarán los datos de un primer contacto con usuarios.\\
Para eso hemos creado un guión de evaluación que fijará unos objetivos y unas preguntas de investigación, así como marcar las pautas para la realización de las pruebas. \\
Después de la realización de las pruebas se procederá a hacer una análisis que muestre las posibles brechas encontradas, basándonos en la experiencia de los usuarios. \\
\section{Objetivos y preguntas de investigación}
Los objetivos y preguntas de investigación tienen como finalidad tratar de comprender la experiencia de los usuarios reales al interactuar con nuestra herramienta por primera vez. Buscamos identificar cualquier punto donde puedan surgir dificultades, confusión o áreas donde la herramienta podría ser más efectiva. Para guiar esta exploración, hemos definido los siguientes objetivos y preguntas de investigación:

\subsection{¿Resulta intuitiva y clara la herramienta para el usuario?}
Lo más importante de la herramienta es que sea clara para los usuarios que van a usarla, ya que sin eso, aunque la herramienta tenga un buen funcionamiento, la utilidad de esta sigue siendo nula.\\
Nuestro primer objetivo se centra en evaluar si los usuarios perciben la interfaz y los flujos de trabajo de manera natural y sin ambigüedades.\\

Preguntas de investigación relacionadas con el objetivo:
\subsubsection{¿El usuario entiende fácilmente cómo añadir funcionalidad al enemigo?}
Esta pregunta de investigación pretende comprobar, si el usuario es capaz de comprender el funcionamiento base de los enemigos, es decir, si comprende como utilizar la máquina de estados o como se añaden estados nuevos, transiciones, Damage Emitters y actuadores.\\

\subsubsection{¿Las opciones de comportamiento ofrecidas son comprendidas en su totalidad?}
Aquí, nuestro interés reside en la claridad con la que se presentan y se entienden los distintos actuadores disponibles. Buscamos identificar cualquier punto de confusión en su aplicación, analizando si la funcionalidad y las explicaciones proporcionadas para cada uno son suficientes para su correcta utilización.\\

\subsubsection{¿La interfaz comunica correctamente el propósito de cada estado y transición?}
La pregunta de investigación pretende conocer si el diseñador es capaz de saber en todo momento en que estado está el enemigo y por tanto que comportamiento cabría esperar.\\

\subsubsection{¿La documentación aclara y detalla la configuración de las funcionalidades?}
Buscamos evaluar si la documentación adjunta proporciona la información necesaria para que el usuario pueda configurar correctamente los enemigos, aprovechando al máximo las funcionalidades implementadas. \\

\subsection{¿El sistema de creación de enemigos demuestra ser funcional?}
Este objetivo busca validar la operatividad y la utilidad de la herramienta.\\

Preguntas de investigación relacionadas con el objetivo:
\subsubsection{¿La herramienta agiliza y simplifica el proceso de diseño de enemigos?}
Esta pregunta se enfoca en la eficiencia que aporta la herramienta al flujo de trabajo del diseñador de videojuegos. ¿Facilita la creación de enemigos?\\

\subsubsection{¿El comportamiento de los enemigos generados se corresponde con lo esperado en el videojuego?}
En este punto, la atención se centra en la fidelidad con la que la lógica definida en la herramienta se manifiesta en el entorno de juego. ¿Actúan los enemigos según lo previsto? ¿Se activan sus acciones en las circunstancias correctas? Identificar cualquier desviación entre el diseño y la ejecución es crucial.\\

\subsubsection{¿La herramienta permite la adaptación y expansión sencilla de los enemigos?}
Queremos evaluar si es posible modificar o ampliar las capacidades de los enemigos creados mediante la adición de nuevos comportamientos o la alteración de parámetros existentes, sin generar complejidad innecesaria.

\section{Diseño de la Evaluación}
En esta sección se detalla la planificación y la estructura de las pruebas con usuarios.

\subsection{Audiencia Objetivo}
\begin{itemize}
\item Edad: mayor de 18 años
\item Genero: No relevante
\item Extracto Sociocultural: El público objetivo se centra en perfiles que tengan un extracto sociocultural relacionado con el diseño de videojuegos. Esto incluye:
\begin{itemize}
\item Estudiantes actuales o pasados de diseño de videojuegos.
\item Profesionales que se dediquen o tengan interés en la creación de enemigos.
\item Personas que sin poseer estudios específicos demuestren tener interés en la creación de entidades y sus comportamientos. 
\end{itemize}

\item Habilidades Requeridas:  Se espera que los participantes en la evaluación cuenten con conocimientos previos en diseño de personajes, conceptos básicos del funcionamiento de Unity y comprensión básica de maquinas de estado.
\end{itemize}


\subsection{Duración y Entorno de Realización}
Cada sesión de prueba se extenderá durante unos 60 minutos. Este lapso lo dedicaremos primero a una breve charla introductoria sobre la prueba. Luego, se pedirá que se explore el manual y realicen los ejemplos prácticos. Finalmente se realizará una entrevista que proporcionará feedback adicional.\\

Las sesiones se realizarán en entornos controlados y tranquilos donde el usuario no tenga distracciones. Se dispondrá un ordenador con la herramienta ya instalada y lista para su uso, así como teclado y ratón estándar. Al haber simultaneidad de pruebas con varios usuarios, para evitar la influencia entre ellos se procurará situarlos separados entre ellos y en caso de que no fuese posible, pedirles la mínima interacción posible entre ellos.

\subsection{Descripción de Tareas del Probador}
El probador deberá realizar las siguientes actividades:

\begin{itemize}
\item Lectura del manual (sin ejemplos de uso). Estimación: 10 minutos.
\item Ejecución de ejemplos de uso. Estimación: 30 minutos.
\item Creación libre de un enemigo. Estimación 10 minutos
\item Realización del cuestionario. Estimación 5 minutos.
\end{itemize}

Las tareas se presentarán de forma secuencial, permitiendo al usuario familiarizarse gradualmente con las funcionalidades de la herramienta. 

\subsection{Instrucciones Iniciales}

Antes de comenzar las pruebas se agradecerá la participación de los usuarios. Se dejara claro al inicio que no es una critica hacia ellos y que solo estamos evaluando la funcionalidad de la herramienta y en ningún caso juzgándoles, 
además, se le indicará las diferentes partes que consta la evaluación.
Guion: 
Buenos días a todos y muchísimas gracias por participar en nuestras pruebas de evaluación. 
Como suponemos que ya todos sabéis, vais a probar una herramienta que sirve para diseñar enemigos en dos dimensiones en Unity. 
Tenéis abierto un proyecto con la herramienta ya cargada. Además hay un manual de uso en el escritorio. 
Por favor leer el manual y hacer los ejemplos de uso. Cuando terminéis probar un poco la herramienta por vuestra cuenta. Os avisaremos cuando abramos el cuestionario posterior a la prueba.

Guion de la Sesión:

"Buenos días a todos y y muchísimas gracias por la colaboración en estas pruebas de evaluación.

Como suponemos que ya todos sabéis, vais a probar una herramienta que sirve para diseñar enemigos en dos dimensiones en Unity.

En los ordenadores tenéis ya un proyecto con la herramienta abierta, así como un manual de usuario detallado en el escritorio.

Por favor leer el manual y hacer de los ejemplos prácticos que se describen dentro de el. Una vez familiarizados con la herramienta, probarla creando distintos enemigos que se os ocurran.
Por último abriremos un breve cuestionario al final para su cumplimentación."
\subsection{Comportamiento del Investigador}

\subsection{Diseño de la evaluación}
 \subsubsection{Observación}
\subsubsection{Cuestionario}
Para conocer la opinión de los usuarios sobre la facilidad de uso del sistema que hemos desarrollado, decidimos emplear el cuestionario SUS (System Usability Scale) de Brooke (1996). 

El SUS consta de diez preguntas en forma de afirmación. Los participantes deben indicar su grado de acuerdo o desacuerdo por medio de una escala Likert de cinco puntos, donde 1 representa totalmente en desacuerdo y 5 representa totalmente de acuerdo. Los ítems que componen el cuestionario son los siguientes:

\begin{enumerate}
    \item Creo que me gustaría usar este sistema con frecuencia.
    \item Encontré el sistema innecesariamente complejo.
    \item Pensé que el sistema era fácil de usar.
    \item Creo que necesitaría la ayuda de una persona técnica para poder usar este sistema.
    \item Encontré que las diversas funciones de este sistema estaban bien integradas.
    \item Pensé que había demasiada inconsistencia en este sistema.
    \item Imaginaría que la mayoría de la gente aprendería a usar este sistema muy rápidamente.
    \item Encontré el sistema muy engorroso de usar.
    \item Me sentí muy confiado al usar el sistema.
    \item Necesité aprender muchas cosas antes de poder usar este sistema.
\end{enumerate}

\subsubsection{Entrevista}
\subsection{Preparación de la Ejecución}
Antes de llevar a cabo las pruebas con usuarios, se realizarán las siguientes acciones de preparación:
\begin{itemize}
\item Preparación de la herramienta (abierta y funcional).
\item Dejar abierta en los ordenadores que se usen para las pruebas la herramienta y el manual.
\item Comprobar el uso de las grabadoras de voz para la entrevista.
\item Tener el guion al alcance para revisar los puntos durante la ejecución.
\end{itemize}
