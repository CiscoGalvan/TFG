\chapter*{Introduction}
\label{cap:introduction}

\chapterquote{Los seres humanos no nacen para siempre el día en que sus madres los alumbran, sino que la vida los obliga a parirse a sí mismos una y otra vez}{Gabriel García Márquez}

\section{Motivation}
Over the years, video games have undergone a remarkable evolution, transforming into more complex entities. In parallel, enemies have followed a similar trajectory. Within the specific context of two-dimensional platformer videogames, enemies are more than just an obstacle for the player; they are key to showcasing the essence of the game. Designing enemies, especially in the aforementioned type of video games, is an increasingly complex task. It is not limited to giving them a certain appearance but requires them to possess unique behaviors and characteristics. Consequently, the person responsible for this task must have certain multidisciplinary knowledge (art, design, programming, ...).
In recent years, tools aimed at significantly simplifying the workflow of designers have emerged. However, a limited proportion of these focus specifically on this workspace. The purpose of these tools lies in facilitating the work of designers, allowing them, even without programming proficiency, the ability to generate enemies with complete functionalities.

\section{Objectives}
The main objective of this project is the design and development of a framework for the \textit{Unity} game engine that simplifies and streamlines the process of creating enemies in 2D platformer games. This framework defines a modular structure of components and behaviors based on the analysis of common enemy types in such games, with the goal of completely separating the roles of programming and design. In this way, it enables individuals without programming knowledge to take on the role of enemy designer.\\

As a practical example of the framework, a functional tool has been developed in Unity that allows these components to be implemented and used in a visual and intuitive manner. The tool includes a catalog of behaviors that is easy to use for anyone, regardless of their programming experience, as well as a user manual that clearly explains each component, installation steps, and usage examples.\\

To carry out this development, a structured work plan has been followed, ranging from an analysis of the development environment and a review of existing enemies, to the implementation of the tool, its validation with users, and analysis of the results obtained.\\


\section{Work Plan}
To carry out this work, the Agile Scrum methodology has been followed. This methodology allows the creation of a workflow focused on iteration and continuous improvement, ensuring efficient development progress and possible adaptations to problems detected during the process.
The work will be divided into four blocks: research and planning, memory development, tool development, and user testing.
Each block will in turn be divided into subsections explained below.
\begin{itemize}
    \item  Research and planning:
	\begin{itemize}
	    \item  Problem study: This initial phase will involve a study of the state of the art, focusing on the role of enemies in video games, their importance in gameplay, and the different techniques used for their design and behavior.
	    \item Tool selection and study: This phase will involve a comparative analysis of different techniques and game engines, evaluating their advantages and disadvantages, as well as a study of their operation and architectures.
	    \item Tool design: In this stage, the architecture of the proposed tool will be defined, describing the techniques used, operation schemes, and organization of main elements.
	\end{itemize}
  \item Memory development:
	\begin{itemize}
	    \item  Initial drafting: In this phase of the work, the initial drafting of the contents will proceed, covering all the points specified in the index.
	    \item Review and correction: Once the initial drafting is completed, the necessary corrections will be made after thoroughly reviewing the document.
	    \item Conclusions and future work: After finalizing the developments and user tests, the conclusions obtained based on the results will be written, and the possible steps to follow in the future will be detailed.
	\end{itemize}
    \item Tool development:
	\begin{itemize}
	    \item  Implementation of main functionalities: In this stage, the main functionalities of basic movements will be implemented, including integration with sensors and emitters, allowing interaction between them.
	    \item Implementation of visual aids: Visual aids will be developed to serve as references for designers, including graphic elements that facilitate the understanding of behaviors.
	    \item Testing and debugging: An iterative testing process will be carried out to ensure the functionality of the tool, correcting errors detected during its implementation.
	\end{itemize}
    \item User testing:
	\begin{itemize}
	    \item First phase of testing: Tests will be carried out with users who have not tried the tool before, following a test plan specified in the \hyperref[cap:evaluacionConUsuarios]{user evaluation} section. The tests will focus on: detecting possible errors in the main functionalities, validating functionality, and evaluating usability and clarity.
	    \item Second phase of testing: After implementing improvements based on feedback from the first test, a second verification test will be carried out.
	    \item Correction and results: After analyzing the results of each test phase, the errors and difficulties encountered will be documented. Following this, the necessary corrections will be implemented to improve the results.
	\end{itemize}

\end{itemize}
