\setcounter{secnumdepth}{3} %para tener una profundidad más en las enumeraciones
\chapter{Implementaci\'on}
\label{cap:implementacion}
En el Capítulo 3 se abordó la descripción de la herramienta sin entrar en detalles de como funcionaba esta por debajo, una visión general de lo que se iba a ofrecer, como la organización de los componentes, los tipos de daño o los diferentes tipos de actuadores. En este capítulo se va a tratar en profundidad la implementación de estos componentes, hablando de cómo funcionan, cómo se pueden personalizar y como los distintos componentes interactúan entre ellos.


\section{Tecnología utilizada}
Este proyecto ha sido desarrollado íntegramente con el motor de videojuegos Unity, mencionado en el Capítulo 2 de este trabajo.
La versión escogida para desarrollar la herramientas es la 2022.3.18f1, por lo que no podemos garantizar que la herramienta funcione en versiones anteriores a la mencionada y en el caso de la versiones posteriores debería funcionar sin ningún problema a no ser que la API básica de Unity cambie en un futuro.\\
\comp{En trabajo futuro podemos poner que en caso de que se cambie la API básica necesitaría algo de mantenimiento básico.}

Unity es una herramienta muy versátil, la cúal se adapta muy bien a un gran rango de aplicaciones diferentes entre sí, desde entornos simples en dos dimensiones hasta entornos mucho más complejos en tres dimensiones, incluso en realidad virtual o realidad aumentada.
Unity surge de la idea de acercar el desarrollo de videojuegos a segmentos de la población que se podrían ver abrumados por la necesidad de entender de programación para realizar sus proyectos ya sean estos profesionales o amateurs.Este motor de videojuegos ofrece soporte para varios lenguajes de programación a través de su sistema de \textit{plugins}, de manera nativa Unity nos ofrece los lenguajes C\# y Javascript como principales lenguajes de programación de \textit{scripts}.\\

Unity cuenta con una interfaz de usuario muy gráfica la cúal resulta muy intuitiva dentro de su complejidad. También es importante mencionar la personalización de su interfaz otorgando al usuario la capacidad de distribuir en pantalla las distintas ventanas que componen la interfaz de usuario de la manera más cómoda posible.
Otro elemento muy importante de Unity es el sistema \textit{drag and drop} el cual nos permite construir las escenas de juego de manera muy sencilla, moviendo los objetos en la escena haciendo click y arrastrándolos y también permite asignar scripts a las entidades de juego de la misma manera. Con respecto a la curva de aprendizaje de Unity esta no es muy pronunciada ya que desde Unity como empresa se toman muy enserio el tener una documentación clara y accesible, así como habilitar foros y tutoriales para que sea la propia comunidad de usuarios la que se ayuda así misma.\\


El motivo por el que se escoge Unity sobre los demás motores de videojuegos es esa accesibilidad que ha sido mencionada anteriormente que hace que Unity resulte sencillo de utilizar por cualquier persona. Otro motivo de peso por el que hemos considerado Unity como nuestra opción principal de motor de videojuegos es la cantidad de usuarios que lo usan día a día lo que lo convierte en una muy buena plataforma para poder poner a prueba nuestra herramienta a través de pruenas de usuario para su posterior uso para el gran público. Esa popularidad de Unity da ciertas garantías de que la herramienta será usada y servirá para confeccionar un mínimo de proyectos.
El sistema \textit{drag and drop} facilita mucho el uso de nuestra herramienta y la comprensión de la misma.
Se hará uso de otros elementos de Unity para llevar a cabo esta herramienta como el motor de físicas 2D o las herramientas que tiene el motor para ayudar al usuario a debuggear como puede ser Gizmos.
Unity funciona mediante una arquitectura por componentes, lo que es muy útil para que el programa sea modular, que pueda ser dividido en piezas más pequeñas y que estas piezas sean independientes.\\
\comp{Faltan fotos en este apartado.}

\section{Infraestructura básica}

Para contextualizar la herramienta y para poder realizar la implementación de los componentes se crearán un repertorio de componentes propios de un videojuego 2D que no forman parte de la arquitectura básica de nuestra herramienta.\\
\subsection{PlayerMovement}
La clase \textit{Player Movement} sirve como controlador del jugador. Esta clase es la encargada de gestionar el input del jugador y acorde a este mover al jugador. También es la clase encargada de detectar si el jugador está situado en el suelo, para ello se usa un objeto auxiliar \textit{GroundCheck}. Este se encuentra en la parte inferior del objeto \textit{Player} y utiliza una circunferencia de radio configurable para determinar si colisiona con alguna superficie. Si hay superposición, se considera que el personaje está en el suelo y, por lo tanto, puede saltar.\\
Esta clase permite modificar ciertos valores como la velocidad de movimiento o la potencia de salto.\\

\subsection{Life}
Se ha implementado una clase Life que gestiona los puntos de salud del objeto al que está adjunto y que se encarga de eliminar el objeto en el caso que su vida llegue a 0. Este componente está estrechamente ligado al \textit{DamageSensor} el cual será mencionado más adelante y que detecta si ha habido una colisión con un objeto que aplique daño, esta relación es necesaria ya que para que se detecte el daño que decrementa la salud del objeto se necesita este sensor.
\subsection{PlayerDistanceAttack}
Ataque a distancia del player
\section {FSM}
El comportamiento de la entidad estará encapsulado en una FSM cuya única funcionalidad es la de gestionar el estado actual y comprobar que no ha habido ningún cambio de estado. Esta comprobación se hará al finalizar la actualización del estado actual para así evitar problemas con cambiar de estado en medio de un bucle sin terminar.

\subsection{State}



\section{Actuators}
\subsection{Movement}
Movement contenido a ver si sale guay
\subsubsection{AAAA}
aaaaa
\subsection{Spawner}
Spawner
\section{Sensors and emitters}
Contenido
\subsection{Sensors}
mas contenido
\subsection{Emitters}
muchisimo mas
