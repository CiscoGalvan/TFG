\chapter{Introducción}
\label{cap:introduccion}

\chapterquote{Los seres humanos no nacen para siempre el día en que sus madres los alumbran, sino que la vida los obliga a parirse a sí mismos una y otra vez}{Gabriel García Márquez}

\section{Motivación}
A lo largo de los años, los videojuegos han evolucionado considerablemente, llegando a convertirse en una parte muy importante del entretenimiento de hoy en dia. Desde las sencillas primeras experiencias de las consolas hasta los complejos mundos actuales, los videojuegos han conquistado audiencias de todas las edades y culturas. A lo largo de esta evolución los enemigos han jugado un papel fundamental, haciendo que los videojuegos presentasen desafios que atrapan a los jugadores. 
En el caso de los videojuegos 2D de plataformas, los enemigos son más que una simple oposición del jugador si no que son clave para mostrar la esencia del juego. El Pac-Man no sería lo mismo sin sus fantasmas ni el Super Mario Bross sin sus Goombas. \\
Diseñar enemigos, especialmente en mundos plataforma en los cuales los enemigos tienen un papel tan importante es siempre una tarea compleja. No solo trata de darles cierta apariencia si no que tienen que tener unos comportamientos y características únicas.  Esto implica que la persona encargada de esta tarea tiene que tener ciertos conocimientos en arte, diseño y programación. En la actualidad, este trabajo es realizado por varias personas. Teniendo como mínimo tres puestos: programador, diseñador y artista. \\
En los ultimos años han surgido herramientas que simplifican significativamente el trabajo de los diseñadores, pero muy pocas estan centradas especificamente en el diseño de enemigos, para que no sea necesario que adquiera conocimientos sobre programación. 


\section{Objetivos}
Este trabajo tiene como objetivo principal el diseño y desarrollo de un herramienta en C sharp para el motor de videojuegos Unity, que simplifique y agilice el proceso  de creación de enemigos en mundos 2D de plataformas. La herramienta contará con un catalogo de comportamientos fácil de manejar para cualquier persona independientemente de sus conocimientos de programación. El catálogo estará compuesto por tres categoías diferentes: acciones, sensores y eventos. Estos se definirán más adelante. 

\section{Plan de trabajo}
Aquí se describe el plan de trabajo a seguir para la consecución de los objetivos descritos en el apartado anterior.

