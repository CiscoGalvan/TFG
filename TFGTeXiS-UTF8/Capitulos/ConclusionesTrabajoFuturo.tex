\chapter{Conclusiones y Trabajo Futuro}
\label{cap:conclusiones}

Esta herramienta ha sido enfocada en todo momento al diseño de enemigos en 2D.  La propuesta surge como respuesta a la necesidad de dotar a diseñadores de una solución práctica y accesible que les permita configurar comportamientos complejos sin requerir conocimientos avanzados en programación.\\
A través de una arquitectura basada en máquinas de estado finitas, y complementada con un conjunto de sensores, actuadores y emisores, se ha logrado una solución flexible, accesible y con capacidad de escalabilidad.\\
Durante el desarrollo, se han cubierto aspectos esenciales como el movimiento, el control de daño, la generación de entidades, el tratamiento de colisiones y la vinculación con animaciones. Además, se ha incluido un manual orientado a usuarios no técnicos, lo que refuerza el propósito de accesibilidad de la herramienta.


\section{Trabajo Futuro}

De cara a futuras iteraciones, se han identificado distintas áreas de expansión que permitirán dar a la herramienta más funcionlidades y por tanto versatilidad:\\
\begin{itemize}
  \item \textbf{Mantenimiento continuo:}  Será necesario realizar revisiones periódicas que aseguren la compatibilidad con futuras versiones del motor Unity, en especial ante posibles modificaciones en su API base.
  \item \textbf{Implementación de sensores de sonido:}  Se planteó incorporar un sensor de sonido capaz de detectar estímulos auditivos, lo que ampliaría las posibilidades de interacción de los enemigos dentro del juego.
  \item \textbf{Emisores de sonido:}  Complementarios a los sensores acústicos, estos componentes permitirían emitir señales auditivas que activen comportamientos en otras entidades cercanas, posibilitando nuevas dinámicas de juego.
  \item \textbf{Comportamientos de Steering y Avoidance: }  Se propone integrar técnicas de evasión y navegación consciente del entorno, que otorguen a los enemigos capacidad para evitar obstáculos y desplazarse de manera más realista.
  \item \textbf{Memoria de interacciones previas: }Una funcionalidad avanzada consistiría en permitir a los enemigos almacenar información sobre acciones pasadas del jugador, adaptando sus respuestas en función de patrones reconocidos.
\item \textbf{Coordinación entre enemigos:}   Otra línea de mejora es permitir la comunicación y cooperación entre múltiples enemigos, generando patrones de comportamiento colectivo, tales como persecuciones en grupo o ataques sincronizados.
 \item \textbf{Desarrollo de una interfaz gráfica externa: }  A largo plazo, se sugiere diseñar una interfaz visual independiente que facilite la selección y configuración de componentes mediante una ventana específica, optimizando así la experiencia del usuario.
\end{itemize}

Estas líneas de trabajo reflejan el potencial del sistema como herramienta activa dentro del flujo de desarrollo de videojuegos 2D. La estructura modular y la orientación hacia la escalabilidad permitirán incorporar estas mejoras de forma progresiva, garantizando su utilidad en proyectos reales de diseño de enemigos complejos.