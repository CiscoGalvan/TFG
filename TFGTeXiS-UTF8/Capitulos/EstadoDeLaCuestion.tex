\chapter{Estado de la Cuestión}
\label{cap:estadoDeLaCuestion}
En este capítulo se hará una investigación sobre las técnicas, herramientas y formas de crear inteligencias artificiales para enemigos.
Para ello vamos a comenzar haciendo un recorrido por todo lo relacionado con la inteligencia artificial en general y cómo se usa esta en videojuegos 2D de plataformas, ya sea para crear un NPC, un enemigo o para optimizar alguna función más a bajo nivel del juego.
Se mencionarán además herramientas para conseguir los fines descritos anteriormente y se hablará de algunos motores de videojuegos que han inspirado en cierta manera algunos aspectos de nuetras herramienta. \\
\section{Introducción a la Inteligencia Artificial}

La Inteligencia Artificial (IA) es la capacidad que tiene un sistema o sotware o de realizar tareas diferentes entre sí de manera autónoma aplicando reglas, algoritmos o patrones de aprendizaje automático, simulando así comportamientos propios de la inteligencia humana.
El potencial de la IA hoy día y por lo que está generando tanto furor es su capacidad de autonomía, ya que no solo sigue reglas, si no que tiene la capacidad de tomar decisiones\\
En el ámbito de los videojuegos, la IA se ha hecho paso a base de demostrar su gran capacidad de adaptación a contextos diversos como la adaptación del Alien en \cite{Alien Isolation}, la manera en la que puede generar contenido procedural para juegos RogueLite como Hades(referencia) o ,por último, entrenar una IA para que se acerque lo máximo posible al comportamiento de un humano como en la serie Forza Motosport, usando redes neuronales.\\

\section{Análisis herramientas}

\section{Tecnicas}

Para modelar estos comportamientos vamos a hacer usos de una serie de herramientas y de técnicas que a continuación vamos a explicar en detalle:
\subsection{Maquinas de estado finitas}
\subsection{Behaviour Bricks}
\subsection{Play Maker}
\subsection{Animator}
\section{Motores de videojuegos}
\subsection{Unity}
\subsection{Unreal}
\subsection{Godot}
\subsection{GameMaker}
\subsection{Construct 3}
\section{Conclusiones}
-----------------------------------------------------\\
En el estado de la cuestión es donde aparecen gran parte de las referencias bibliográficas del trabajo. Una de las formas más cómodas de gestionar la bibliografía en {\LaTeX} es utilizando \textbf{bibtex}. Las entradas bibliográficas deben estar en un fichero con extensión \textit{.bib} (con esta plantilla se proporciona el fichero biblio.bib, donde están las entradas referenciadas más abajo). Cada entrada bibliográfica tiene una clave que permite referenciarla desde cualquier parte del texto con los siguiente comandos:

\begin{itemize}
\item Referencia bibliografica con cite: \cite{ldesc2e}
\item Referencia bibliográfica con citep: \citep{notsoshort}
\item Referencia bibliográfica con citet: \citet{latexAPrimer}
\end{itemize}

Es posible citar más de una fuente, como por ejemplo \citep{latexCompanion,LaTeXLamport,texKnuth}

Después, \LaTeX se ocupa de rellenar la sección de bibliografía con las entradas \textbf{que hayan sido citadas} (es decir, no con todas las entradas que hay en el .bib, sino sólo con aquellas que se hayan citado en alguna parte del texto).

Bibtex es un programa separado de latex, pdflatex o cualquier otra cosa que se use para compilar los .tex, de manera que para que se rellene correctamente la sección de bibliografía es necesario compilar primero el trabajo (a veces es necesario compilarlo dos veces), compilar después con bibtex, y volver a compilar otra vez el trabajo (de nuevo, puede ser necesario compilarlo dos veces). 
